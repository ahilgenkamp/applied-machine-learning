\documentclass{article}
\usepackage[utf8]{inputenc}
\usepackage{csquotes}
\usepackage[english]{babel}
\usepackage[natbibapa]{apacite}

\title{Improving the Invoice-to-Cash Process with Customer Payment Predictions \\ CSCI-P556: Applied Machine Learning \\ Fall 2019}
\author{Adam Hilgenkamp (ahilgenk@iu.edu) \\ Kaitlin Pet (kpet@iu.edu) \\ Jashjeet Singh Madan (jsmadan@iu.edu) \\ Surya Soni (susoni@iu.edu)}
\date{September 28, 2019}

\usepackage{graphicx}
\usepackage{geometry}
 \geometry{
 left=35mm,
 right=30mm,
 top=30mm,
 bottom=35mm
 }

%%\usepackage{indentfirst}
\setlength{\parindent}{3em}
\setlength{\parskip}{1em}

\begin{document}

\maketitle

\section{Problem Statement}
According to \citet[p.1]{schwab2016} \enquote{[The Fourth Industrial Revolution] is fundamentally changing the way we live, work, and relate to one another}. One trend of the Fourth Industrial Revolution that is highlighted by Klaus Schwab and the World Economic Forum is the increase of customer expectations.  Customers increasingly expect a personalized and seamless experience, and businesses frequently rely on customer data to deliver that experience \cite[p.53]{schwab2016}. 
\par
An important part of interacting with a business is paying for the product or service. For businesses that send bills to their customers (instead of collecting payment immediately), this is known as the invoice-to-cash process. The invoice-to-cash process consists of invoicing the customer, collecting the payment from the customer and crediting the payment to their account. However, this payment is not always made on time for a variety of reasons. Some companies may not be able to pay and others want to hold onto cash reserves for as long as possible. Being able to predict whether a customer will pay their invoice on time has two main advantages for a business: a) helping the company effectively manage its cash flow expectations for its key stakeholders and b) improving the customer experience by reducing the need for intervention from collectors.
%% Changed wording from service provider to business(es) to make it more broad as these are problems that companies providing a service as well as selling physical products must deal with.  The nomenclature would be business and customer.
\par
First, predicting whether a given customer will pay on time is important in optimizing customer experience in the invoice-to-cash process. Large businesses that sell products or services to other businesses typically invoice and collect payments from customers with some level of automation. Automating the process of creating and sending invoices is fairly straightforward, but the process of "following up" with customers is more labor intensive and time consuming. A one-size-fits-all approach to automating the follow up process would result in either flooding customers who usually pay on time with messages or not enough contact with customers who may not pay on time. With increasing customer expectations in mind, businesses need to find a way to avoid sending excessive messages to a customer who consistently pays on time while still being proactive to ensure payments are collected in a reasonable amount of time. Bucketing customers by their predicted payment behaviors would allow a business to tailor "follow-up" strategies for each type of customer. 
\par
Another challenge for companies is forecasting the cash inflow in a given period and how close they would be to internal targets and external expectations from Wall Street. This forecast is typically calculated using general historical trends and qualitative information from the collections team. The ability to predict a) whether customers will pay their invoice and b) how late customers pay their invoice would give a company a better quantitative prediction on cash flow for a given period.  The result of more accurate predictions allows the managers of the business to make proactive decisions such as paying an invoice early if they expect to collect more money in a period. The combination of greater accuracy and better customer experiences will be necessary for businesses to compete and succeed in the Fourth Industrial Revolution.


\section{Data}
In this paper we will be examining data from a Fortune 500 technology company.  We have been given access to a redacted invoice dataset containing approximately 7.7 million observations and 139 features dating back to January of 2015.  However, not all of the features are relevant or sufficiently populated so we will need to filter down our selection of features to reduce the noise in the data. To supplement the invoice data we have access to other datasets that we can use to add additional features.  Additional datasets include account details which will allow us to bring in attributes such as the size of the customer (based on publicly reported annual revenue) and order details which will allow us to include the products purchased.  
\par
The dataset is redacted in order to remove any information that would identify a specific customer such as company name, address, or phone number.  These fields are instead transformed into numeric IDs that obfuscate the fields but still allow us to use them in our models. 
\par
Each record in the invoice dataset is created when the invoice is sent to the customer. The record initially is filled with information about the invoice such as payment terms, payment type, billing country and the amount due.  The invoice also has additional fields that are updated throughout the process.  As part of our feature selection we will need to remove any of these fields that could "leak" information about the future to the model.  For example, the last contacted date would be updated each time contact is made.  This field would not be populated when a new invoice is observed by the model so the prediction using future information about the process will effect the error rate when the model is used to classify or predict real world data. 
%% Will want to include a data dictionary as a supplemental table.
%% Include some descriptive statistics about the dataset and the data pre-processing 

\section{Research questions}
 When invoices are sent out to customers they are expected to pay by the due date.  Late payments occur when customers either cannot pay or want to hold onto cash reserves and decide to pay late.  The aim of this project is to predict the timing and outcome of the payment, i.e. informing business leaders if they can expect this payment within a certain time period. If the payment timing can be predicted with high accuracy, then the company can make decisions to manage its cash to meet stakeholder expectations and avoid intervention by collectors, hence increasing the overall customer satisfaction. Based on this goal, we have three research questions we would like to address: 
 
\begin{enumerate}

%%    \item What is the effect of a new customer's attributes on the possibilities of whether he/she will pay on time or not? (Clustering and classification) 

%%   \item Can we utilize payment history as an effective feature to enhance prediction for existing customers?
    
%%    Given information on if a customer is likely to pay or not how many days late will they pay (regression)? This will help the service provider predict total income for a certain period.  
    

    \item Can we accurately predict whether a new customer will pay on time (classification)? This will help improve customer experience by bucketing customers into different communication strategies. For example, a 'low risk' customer may only receive one email while a 'high risk' customer would be personally called by a company representative.
    
    \item Given information on if a customer is likely to pay or not how many days late will they pay (regression)? This will help the business better predict total cash inflow for a certain period. 
    
    \item Can we utilize case notes as an effective feature, i.e. can we use NLP techniques to automate analysis of notes instead of reading them on a case-by-case basis?
       
\end{enumerate}

\section{Evaluation Criteria}
We will randomly split the dataset into training and test data. The training data will be split into several sections and models will be tested using cross-validation. Given the fact that our data points vastly outnumber potential features, we are unlikely to have issues with high variance. Therefore, inaccuracy in predictions will likely stem from high bias, i.e. we are more likely to underfit than overfit our data unless there are errors in the cross-validation process.  

Work has also been done using Einstein Prediction builder, which is a Salesforce product that aims to democratize learning algorithms to the business user. It is a UI based AI which assists in building custom queries and formulae. We can compare performance of our model against the Einstein Prediction builder.  

\section{Timeline and Roles}

\begin{center}
    \begin{tabular}{p{3cm}|p{6cm}|p{2cm}|p{2cm}}
    \hline
    Date & Deliverable & Research Question & Members Involved \\
    \hline
    October 11 & Data cleansing done on base invoice dataset & General & Adam \\
    \hline
    October 12-18 & Begin building models and testing on the base dataset & General & All \\
    \hline
    October 18 & Join in additional fields to base invoice dataset & General & Adam \\
    \hline
    October 25 & Finish building minimal working classification model & Question 1 & Adam and Surya \\
    \hline
    October 25 & Finish building minimal working linear regression model & Question 2 & Jashjeet and Kaitlin \\
    \hline
    November 8 & Test whether payment history (time series data) can be incorporated into regression model as predictor & Question 3 & Kaitlin and Adam \\
    \hline
    November 8  & Test whether sentiment analysis can be used to transform notes into predictive feature & Question 4 & Surya and Jashjeet \\
    \hline
    November 24 & Finish Building Models & General & All
    \end{tabular}
\end{center}



\newpage
\bibliographystyle{apacite}
\bibliography{AML_Project_Proposal}
\end{document}\documentclass{article}
\usepackage[utf8]{inputenc}
\usepackage{csquotes}
\usepackage[english]{babel}
\usepackage[natbibapa]{apacite}

\title{Improving the Invoice-to-Cash Process with Customer Payment Predictions \\ CSCI-P556: Applied Machine Learning \\ Fall 2019}
\author{Adam Hilgenkamp (ahilgenk@iu.edu) \\ Kaitlin Pet (kpet@iu.edu) \\ Jashjeet Singh Madan (jsmadan@iu.edu) \\ Surya Soni (susoni@iu.edu)}
\date{September 28, 2019}

\usepackage{graphicx}
\usepackage{geometry}
 \geometry{
 left=35mm,
 right=30mm,
 top=30mm,
 bottom=35mm
 }

%%\usepackage{indentfirst}
\setlength{\parindent}{3em}
\setlength{\parskip}{1em}

\begin{document}

\maketitle

\section{Problem Statement}
According to \citet[p.1]{schwab2016} \enquote{[The Fourth Industrial Revolution] is fundamentally changing the way we live, work, and relate to one another}. One trend of the Fourth Industrial Revolution that is highlighted by Klaus Schwab and the World Economic Forum is the increase of customer expectations.  Customers increasingly expect a personalized and seamless experience, and businesses frequently rely on customer data to deliver that experience \cite[p.53]{schwab2016}. 
\par
An important part of interacting with a business is paying for the product or service. For businesses that send bills to their customers (instead of collecting payment immediately), this is known as the invoice-to-cash process. The invoice-to-cash process consists of invoicing the customer, collecting the payment from the customer and crediting the payment to their account. However, this payment is not always made on time for a variety of reasons. Some companies may not be able to pay and others want to hold onto cash reserves for as long as possible. Being able to predict whether a customer will pay their invoice on time has two main advantages for a business: a) helping the company effectively manage its cash flow expectations for its key stakeholders and b) improving the customer experience by reducing the need for intervention from collectors.
%% Changed wording from service provider to business(es) to make it more broad as these are problems that companies providing a service as well as selling physical products must deal with.  The nomenclature would be business and customer.
\par
First, predicting whether a given customer will pay on time is important in optimizing customer experience in the invoice-to-cash process. Large businesses that sell products or services to other businesses typically invoice and collect payments from customers with some level of automation. Automating the process of creating and sending invoices is fairly straightforward, but the process of "following up" with customers is more labor intensive and time consuming. A one-size-fits-all approach to automating the follow up process would result in either flooding customers who usually pay on time with messages or not enough contact with customers who may not pay on time. With increasing customer expectations in mind, businesses need to find a way to avoid sending excessive messages to a customer who consistently pays on time while still being proactive to ensure payments are collected in a reasonable amount of time. Bucketing customers by their predicted payment behaviors would allow a business to tailor "follow-up" strategies for each type of customer. 
\par
Another challenge for companies is forecasting the cash inflow in a given period and how close they would be to internal targets and external expectations from Wall Street. This forecast is typically calculated using general historical trends and qualitative information from the collections team. The ability to predict a) whether customers will pay their invoice and b) how late customers pay their invoice would give a company a better quantitative prediction on cash flow for a given period.  The result of more accurate predictions allows the managers of the business to make proactive decisions such as paying an invoice early if they expect to collect more money in a period. The combination of greater accuracy and better customer experiences will be necessary for businesses to compete and succeed in the Fourth Industrial Revolution.


\section{Data}
In this paper we will be examining data from a Fortune 500 technology company.  We have been given access to a redacted invoice dataset containing approximately 7.7 million observations and 139 features dating back to January of 2015.  However, not all of the features are relevant or sufficiently populated so we will need to filter down our selection of features to reduce the noise in the data. To supplement the invoice data we have access to other datasets that we can use to add additional features.  Additional datasets include account details which will allow us to bring in attributes such as the size of the customer (based on publicly reported annual revenue) and order details which will allow us to include the products purchased.  
\par
The dataset is redacted in order to remove any information that would identify a specific customer such as company name, address, or phone number.  These fields are instead transformed into numeric IDs that obfuscate the fields but still allow us to use them in our models. 
\par
Each record in the invoice dataset is created when the invoice is sent to the customer. The record initially is filled with information about the invoice such as payment terms, payment type, billing country and the amount due.  The invoice also has additional fields that are updated throughout the process.  As part of our feature selection we will need to remove any of these fields that could "leak" information about the future to the model.  For example, the last contacted date would be updated each time contact is made.  This field would not be populated when a new invoice is observed by the model so the prediction using future information about the process will effect the error rate when the model is used to classify or predict real world data. 
%% Will want to include a data dictionary as a supplemental table.
%% Include some descriptive statistics about the dataset and the data pre-processing 

\section{Research questions}
 When invoices are sent out to customers they are expected to pay by the due date.  Late payments occur when customers either cannot pay or want to hold onto cash reserves and decide to pay late.  The aim of this project is to predict the timing and outcome of the payment, i.e. informing business leaders if they can expect this payment within a certain time period. If the payment timing can be predicted with high accuracy, then the company can make decisions to manage its cash to meet stakeholder expectations and avoid intervention by collectors, hence increasing the overall customer satisfaction. Based on this goal, we have three research questions we would like to address: 
 
\begin{enumerate}

%%    \item What is the effect of a new customer's attributes on the possibilities of whether he/she will pay on time or not? (Clustering and classification) 

%%   \item Can we utilize payment history as an effective feature to enhance prediction for existing customers?
    
%%    Given information on if a customer is likely to pay or not how many days late will they pay (regression)? This will help the service provider predict total income for a certain period.  
    

    \item Can we accurately predict whether a new customer will pay on time (classification)? This will help improve customer experience by bucketing customers into different communication strategies. For example, a 'low risk' customer may only receive one email while a 'high risk' customer would be personally called by a company representative.
    
    \item Given information on if a customer is likely to pay or not how many days late will they pay (regression)? This will help the business better predict total cash inflow for a certain period. 
    
    \item Can we utilize case notes as an effective feature, i.e. can we use NLP techniques to automate analysis of notes instead of reading them on a case-by-case basis?
       
\end{enumerate}

\section{Evaluation Criteria}
We will randomly split the dataset into training and test data. The training data will be split into several sections and models will be tested using cross-validation. Given the fact that our data points vastly outnumber potential features, we are unlikely to have issues with high variance. Therefore, inaccuracy in predictions will likely stem from high bias, i.e. we are more likely to underfit than overfit our data unless there are errors in the cross-validation process.  

Work has also been done using Einstein Prediction builder, which is a Salesforce product that aims to democratize learning algorithms to the business user. It is a UI based AI which assists in building custom queries and formulae. We can compare performance of our model against the Einstein Prediction builder.  

\section{Timeline and Roles}

\begin{center}
    \begin{tabular}{p{3cm}|p{6cm}|p{2cm}|p{2cm}}
    \hline
    Date & Deliverable & Research Question & Members Involved \\
    \hline
    October 11 & Data cleansing done on base invoice dataset & General & Adam \\
    \hline
    October 12-18 & Begin building models and testing on the base dataset & General & All \\
    \hline
    October 18 & Join in additional fields to base invoice dataset & General & Adam \\
    \hline
    October 25 & Finish building minimal working classification model & Question 1 & Adam and Surya \\
    \hline
    October 25 & Finish building minimal working linear regression model & Question 2 & Jashjeet and Kaitlin \\
    \hline
    November 8 & Test whether payment history (time series data) can be incorporated into regression model as predictor & Question 3 & Kaitlin and Adam \\
    \hline
    November 8  & Test whether sentiment analysis can be used to transform notes into predictive feature & Question 4 & Surya and Jashjeet \\
    \hline
    November 24 & Finish Building Models & General & All
    \end{tabular}
\end{center}



\newpage
\bibliographystyle{apacite}
\bibliography{AML_Project_Proposal}
\end{document}
